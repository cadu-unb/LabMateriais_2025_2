\documentclass[conference]{IEEEtran}

% --- Pacotes ---
\usepackage[utf8]{inputenc}
\usepackage[T1]{fontenc}
\usepackage[brazil]{babel}
\usepackage{amsmath,amsfonts,amssymb}
\usepackage{graphicx}
\usepackage{float}
\usepackage{booktabs}
\usepackage{siunitx}
\usepackage{hyperref}

\sisetup{separate-uncertainty=true}

% --- Dados do relatório ---
\title{Estudo Experimental do Efeito Fotoelétrico}

\author{
\IEEEauthorblockN{Gabriel Dantas de Moraes Almeida (231013064)}
\IEEEauthorblockA{Universidade de Brasília (UnB)\\
Email: gabriel.almeida@aluno.unb.br}
\and
\IEEEauthorblockN{Iuri Clemente Rezende (231025619)}
\IEEEauthorblockA{Universidade de Brasília (UnB)\\
Email: iuri.rezende@aluno.unb.br}
\and
\IEEEauthorblockN{Kelvin Evangelista Neiva (231013144)}
\IEEEauthorblockA{Universidade de Brasília (UnB)\\
Email: kelvin.neiva@aluno.unb.br}
}

\date{27 de maio de 2025}

\begin{document}
\maketitle

\begin{abstract}
Este trabalho apresenta o estudo do efeito fotoelétrico, fenômeno no qual elétrons são emitidos de uma superfície metálica ao ser iluminada por radiação eletromagnética de frequência suficientemente alta. O objetivo foi determinar a constante de Planck, a função trabalho da fotocélula e a frequência de corte da radiação incidente. Foram realizadas medições utilizando uma lâmpada de mercúrio, rede de difração, filtros e fotocélula. Os resultados obtidos indicaram a correta ordem de grandeza da constante de Planck e permitiram discutir limitações experimentais.
\end{abstract}

\section{Introdução}
O efeito fotoelétrico consiste na emissão de elétrons por um material metálico exposto à radiação eletromagnética com frequência superior a um valor mínimo denominado \textit{frequência de corte}. Esse fenômeno foi fundamental para o desenvolvimento da mecânica quântica, pois não pode ser explicado pela teoria clássica da luz.  

O presente experimento busca: 
\begin{itemize}
    \item Determinar a constante de Planck ($h$);
    \item Calcular a função trabalho ($\omega_0$) da fotocélula utilizada;
    \item Identificar a frequência de corte ($\nu_0$).
\end{itemize}

\section{Materiais e Métodos}
O experimento foi realizado em laboratório utilizando os seguintes equipamentos:
\begin{itemize}
    \item Fotocélula;
    \item Rede de difração (600 linhas/mm);
    \item Filtros: \SI{580}{nm} e \SI{525}{nm};
    \item Fenda de abertura ajustável;
    \item Lente convergente ($f = +\SI{100}{mm}$);
    \item Lâmpada de mercúrio;
    \item Trilho de sustentação;
    \item Amplificador universal de medição;
    \item Multímetro digital.
\end{itemize}

A lâmpada de mercúrio foi posicionada em uma extremidade do trilho e a fotocélula na outra. A fenda foi colocada a \SI{9}{cm} da lâmpada e a lente a aproximadamente \SI{20}{cm}.  
O amplificador foi configurado como eletrômetro ($R_{in} > 10^{13}\ \Omega$) com ganho 100. O voltímetro foi ajustado para escala de \SI{2}{V} DC.  

\subsection{Procedimentos}
\begin{enumerate}
    \item Selecionar a faixa monocromática desejada com a rede de difração;
    \item Com a janela da fotocélula fechada, verificar o zero do amplificador;
    \item Abrir a janela, aguardar estabilização e medir a tensão;
    \item Fechar a janela.
\end{enumerate}

\subsection{Cuidados experimentais}
\begin{itemize}
    \item Os equipamentos foram ligados 15 minutos antes das medidas;
    \item A lâmpada não deve ser desligada antes do resfriamento (\SI{10}{min});
    \item Não tocar no bulbo ($T > 100^\circ$C);
    \item Não olhar diretamente para a lâmpada (radiação UV nociva);
    \item A janela da fotocélula permaneceu fechada fora das medições.
\end{itemize}

\section{Resultados Experimentais}
A relação fundamental é:
\begin{equation}
\lambda = d \cdot \sin(\theta), \qquad f = \frac{c}{\lambda} = \frac{c}{d \cdot \sin(\theta)}
\end{equation}

\subsection{Medições obtidas}
\begin{table}[H]
\centering
\begin{tabular}{cccc}
\toprule
Medida & $\theta$ [º] & $f$ [$10^{14}$ Hz] & $V$ [V] \\
\midrule
azul1 & 13 & 7.996 & 0.812 \\
azul2 & 15 & 6.949 & 0.663 \\
azul3 & 16 & 6.525 & 0.660 \\
verde & 20 & 5.259 & 0.398 \\
laranja & 21 & 5.019 & 0.333 \\
azul1 & 12 & 8.651 & 0.826 \\
azul2 & 15 & 6.949 & 0.690 \\
azul3 & 16 & 6.525 & 0.686 \\
verde & 20 & 5.259 & 0.406 \\
laranja & 21 & 5.019 & 0.338 \\
azul1 & 13 & 7.996 & 0.860 \\
azul2 & 15 & 6.949 & 0.657 \\
azul3 & 16 & 6.525 & 0.670 \\
verde & 20 & 5.259 & 0.408 \\
laranja & 21 & 5.019 & 0.340 \\
\bottomrule
\end{tabular}
\caption{Medições realizadas com diferentes filtros.}
\end{table}

Após a média dos valores:
\begin{table}[H]
\centering
\begin{tabular}{cc}
\toprule
Frequência Média [$10^{14}$ Hz] & Tensão Média [V] \\
\midrule
8.214 & 0.833 \\
6.949 & 0.670 \\
6.525 & 0.672 \\
5.259 & 0.404 \\
5.019 & 0.337 \\
\bottomrule
\end{tabular}
\caption{Valores médios de frequência e tensão obtidos.}
\end{table}

\subsection{Análise}
A energia cinética máxima é:
\begin{equation}
K_{max} = h \cdot f - \omega_0
\end{equation}
e o potencial de retardo é dado por:
\begin{equation}
V_0(f) = \frac{h}{e} \cdot f - \frac{\omega_0}{e}
\end{equation}

Constantes obtidas:
\[
h_{exp} = 2.48 \times 10^{-34}\ \text{J·s}, \quad
\omega_0 = 6.51 \times 10^{-20}\ \text{J} = 0.407\ \text{eV}
\]

Comparação:
\[
h_{real} = 6.62 \times 10^{-34}\ \text{J·s}
\]

Frequência de corte:
\[
\nu_0 = 2.628 \times 10^{14}\ \text{Hz}
\]

\section{Discussão}
O valor obtido para $h$ foi da mesma ordem de grandeza do esperado, atendendo ao requisito experimental. No entanto, a função trabalho não corresponde a nenhum material conhecido, sugerindo erros experimentais.  

O efeito fotoelétrico só ocorre para frequências acima da frequência de corte, confirmando o modelo quântico da luz de Einstein. A teoria clássica não explica a ausência de emissão para intensidades altas, mas frequências baixas.

\section{Conclusão}
O experimento permitiu estimar a constante de Planck, ainda que com erro relativo elevado, e compreender conceitos como função trabalho, frequência de corte e potencial de retardo. Observou-se a limitação dos equipamentos e as dificuldades na caracterização exata do material da fotocélula.  

\section*{Referências}
\begin{thebibliography}{00}
\bibitem{ceschin} A. Ceschin, \textit{Materiais Elétricos e Magnéticos}. Brasília, DF, 226 p.
\bibitem{crc} W. M. Haynes (ed.), \textit{CRC Handbook of Chemistry and Physics}, 97ª ed. Boca Raton: CRC Press, 2016.
\end{thebibliography}

\end{document}
