\documentclass[conference]{IEEEtran}

% --- Pacotes ---
\usepackage[utf8]{inputenc}
\usepackage[T1]{fontenc}
\usepackage[brazil]{babel}
\usepackage{amsmath,amsfonts,amssymb}
\usepackage{graphicx}
\usepackage{float}
\usepackage{booktabs}
\usepackage{siunitx}
\usepackage{hyperref}
\usepackage[dvipsnames]{xcolor}
\usepackage{mdframed}

% --- Configurações ---
\sisetup{
    separate-uncertainty=true,
    output-decimal-marker={,}
}
\definecolor{primaryBlue}{HTML}{003366}
\definecolor{accentTeal}{HTML}{00629B}
\definecolor{lightGray}{HTML}{F8F8F8}

\hypersetup{
    colorlinks=true,
    linkcolor=primaryBlue,
    citecolor=accentTeal,
    urlcolor=accentTeal,
    pdftitle={Experimento 2 - Efeito Fotoelétrico},
    pdfauthor={Larissa Simões, Thiago Ferreira}
}

% --- Ambiente de destaque ---
\newmdenv[
  linecolor=accentTeal,
  linewidth=1.5pt,
  roundcorner=7pt,
  backgroundcolor=lightGray,
  innertopmargin=6pt,
  innerbottommargin=6pt,
  innerleftmargin=8pt,
  innerrightmargin=8pt,
  skipabove=8pt,
  skipbelow=8pt
]{highlightbox}

% --- Página ---
\pagestyle{empty}
\setlength{\parindent}{1em}
\setlength{\parskip}{4pt}
\renewcommand{\baselinestretch}{1.05}\selectfont

% --- Título ---
\title{\color{primaryBlue}Experimento 2: Estudo Dirigido sobre Efeito Fotoelétrico}
\author{
\IEEEauthorblockN{Larissa Simões – 232028230}
\IEEEauthorblockA{Turma 01}
\and
\IEEEauthorblockN{Thiago Ferreira – 231025717}
\IEEEauthorblockA{Turma 01}
}

\begin{document}
\maketitle

\begin{abstract}
Este relatório aborda o estudo do efeito fotoelétrico, fenômeno no qual elétrons são emitidos de uma superfície metálica quando exposta à radiação eletromagnética de frequência suficientemente elevada. O propósito do experimento foi estimar a constante de Planck, a função trabalho da fotocélula e a frequência de corte da radiação incidente. Para isso, foram utilizados uma lâmpada de mercúrio, rede de difração, filtros e uma fotocélula. Os resultados obtidos possibilitaram a análise da ordem de grandeza da constante de Planck e a discussão de limitações inerentes ao processo experimental.
\end{abstract}

\begin{IEEEkeywords}
efeito fotoelétrico, constante de Planck, função trabalho, frequência de corte
\end{IEEEkeywords}

\section{Introdução}
O efeito fotoelétrico corresponde à liberação de elétrons de um material metálico quando iluminado por radiação eletromagnética cuja frequência ultrapassa um valor mínimo, conhecido como frequência de corte. Esse fenômeno foi fundamental para o avanço da mecânica quântica, visto que não pode ser descrito adequadamente pelos modelos da física clássica.

\subsection{Objetivos do Experimento:}

\begin{itemize}
    \item Determinar experimentalmente a constante de Planck ($h$);
    \item Calcular a função trabalho ($\phi$) da fotocélula utilizada;
    \item Identificar a frequência de corte ($f_0$) da radiação.
\end{itemize}

\section{Materiais e Montagem}
\begin{itemize}
    \item Fotocélula;
    \item Rede de difração (\SI{600}{linhas/mm});
    \item Filtros de cor: \SI{580}{\nano\meter} e \SI{525}{\nano\meter};
    \item Fenda de abertura ajustável;
    \item Lente convergente ($f = +\SI{100}{\milli\meter}$);
    \item Lâmpada de mercúrio;
    \item Trilho de sustentação;
    \item Amplificador universal de medição (configurado como eletrômetro);
    \item Multímetro digital.
\end{itemize}

\begin{figure}[!htbp]
    \centering
    \includegraphics[width=0.5\linewidth]{16279478-d25c-4589-a3d5-449dc32c9104.JPG}
    \caption{Montagem experimental utilizada para o estudo do efeito fotoelétrico.}
    \label{fig:montagem}
\end{figure}

\section{Procedimentos Experimentais}
Inicialmente, a lâmpada foi posicionada em uma das extremidades do trilho e a fotocélula na outra. A fenda foi ajustada a uma distância de \SI{9}{\centi\meter} da fonte, e a lente, a aproximadamente \SI{20}{\centi\meter} da fenda. O voltímetro foi configurado na escala de \SI{2}{\volt} DC. Em seguida, selecionou-se o feixe monocromático por meio da rede de difração. Com a janela da fotocélula inicialmente fechada, realizou-se o ajuste de zero no amplificador. Após a estabilização, a janela foi aberta para registrar a tensão correspondente, sendo fechada ao final de cada medição.

Algumas precauções foram necessárias: o sistema foi ligado \SI{15}{minutos} antes das medições; manteve-se distância do bulbo devido à temperatura elevada (acima de \SI{100}{\celsius}); e evitou-se a observação direta da lâmpada devido à emissão de radiação ultravioleta.

\section{Resultados Experimentais}

A relação fundamental para a frequência ($f$) em função do ângulo de difração ($\theta$), do espaçamento da rede ($d$) e da velocidade da luz ($c$) é dada por:

\begin{align}
\lambda &= d \cdot \sin(\theta) \\
\\
f &= \frac{c}{\lambda} = \frac{c}{d \cdot \sin(\theta)}
\end{align}

\subsection{Medições obtidas}
\begin{table}[H]
\centering
\caption{Medições de tensão de parada para diferentes frequências.}
\label{tab:medicoes}
\begin{tabular}{cccc}
 \toprule
\textbf{Medida} & \textbf{$\theta$ / \si{\degree}} & \textbf{$f$ / \SI{e14}{\hertz}} & \textbf{$V$ / \si{\volt}} \\
\midrule
Azul 1 & 11 & 8,218 & 1,64 \\
Azul 2 & 13 & 6,949 & 1,33 \\
Roxo & 18 & 6,735 & 1,28 \\
Verde & 20 & 5,368 & 0,96 \\
Laranja & 21 & 5,125 & 0,91 \\
\bottomrule
\end{tabular}
\end{table}

\subsection{Análise}
A energia cinética máxima ($K_{max}$) dos fotoelétrons é linearmente dependente da frequência ($f$) da luz incidente, conforme a equação de Einstein:
\begin{equation}
K_{max} = h \cdot f - \phi
\end{equation}
onde $h$ é a constante de Planck e $\phi$ é a função trabalho do material. A energia $K_{max}$ pode ser determinada pelo potencial de retardo ($V_0$), tal que $K_{max} = e \cdot V_0$. Portanto, a relação para o potencial de retardo é:
\begin{equation}
V_0(f) = \frac{h}{e} \cdot f - \frac{\phi}{e}
\end{equation}

A partir de um ajuste linear dos dados da Tabela \ref{tab:medicoes}, foram obtidos os seguintes valores experimentais:

\begin{figure}[!htbp]
\centering
\includegraphics[width=0.75\linewidth]{PHOTO-2025-09-29-09-37-49.jpg}
\caption{Gráfico de pontos com regressão linear.}
\label{fig:placeholder}
\end{figure}

\begin{itemize}
\item \textbf{Constante de Planck experimental:}
\[ h_{exp} = \SI{3.783e-34}{\joule\second} \]
\item \textbf{Função trabalho:}
\[ \phi = \SI{0.306}{\electronvolt} \]
\end{itemize}

Para fins de comparação, o valor teórico da constante de Planck é:
\[ h_{real} = \SI{6.626e-34}{\joule\second} \]

A frequência de corte ($f_0$), que é a frequência mínima para a ocorrência do efeito, foi calculada como:
\[ f_0 = \frac{\phi}{h_{exp}} = \SI{1.295e14}{\hertz} \]

\section{Discussão}
\subsection{Questões Propostas:}

\begin{enumerate}
    \item \textbf{Compare o valor da constante de Planck obtido no experimento com o valor encontrado nos livros. Comente a precisão do experimento destacando as possíveis fontes de erro.}\\

    - Valor da constante de Planck obtido experimentalmente:
    \textbf{\[ h_{exp} = \SI{3.783e-34}{\joule\second} \]}
    - O valor esperado:
    \[ h_{real} = \SI{6.626e-34}{\joule\second} \]

    O valor encontrado se aproxima em magnitude do valor presente na literatura.\\
    
    \item \textbf{Explique como o potencial de retardo pode afetar o valor de h obtido no experimento.}\\

    A luz em alta frequência incide sobre o cátodo e, portanto, ocorre o efeito fotoelétrico. Os fótons em forma de onda, fornecem energia para os elétrons do material. Se a energia da onda for maior que a função trabalho, elétrons sâo ejetados com k$_{max}$.\\

    \item \textbf{Podemos utilizar o valor da função trabalho encontrado para caracterizar o material do catodo? Explique porque.}\\

    Podemos utilizar a função trabalho para caracterizar o material, dado que a função trabalho é a energia mínima necessária para remover um elétron da superfície de um material específico. Dado que os fotoelétrons em direção ao cátodo, criam uma corrente elétrica dentro da célula.\\

    \item \textbf{Explique utilizando conceitos de física quântica, porque o fenômeno da emissão fotoelétrica não ocorre para frequências abaixo da frequência de corte. É possível explicar o experimento utilizando apenas a física clássica? Explique.}\\

    A física quântica fundamenta o efeito fotoelétrico de forma em que a luz não é uma onda contínua, mas sim pacotes discretos de energia, os fótons. Seguindo que $E$ $=$ $hf$, que implica diretamente na ejeção de elétrons, dado que é definido k$_{max}$ para ejetar os elétrons.
    No entanto, ao observar a física clássica, conclui-se que não é possível explicar o fenômeno, dado que a emissão de elétrons depende, exclusivamente, da intensidade da luz.\\

    \item \textbf{Explique o que ocorre na célula fotoelétrica desde o momento em que é descarregada, até o momento que a leitura do multímetro é máxima.} \\

    A luz atinge a célula descarregada, elétrons são ejetados do cátodo e viajam para o ânodo, gerando corrente, que faz com que haja diferença de potencial entre as placas. A tensão possui um campo elétrico, que atua como freio, dificultando a passagem de elétrons subsequentes. À medida que o potencial de frenagem aumenta, a corrente diminui até que a leitura do multímetro atinge seu valor máximo. \\

    \item \textbf{Os elétrons que absorvem fótons com a frequência de corte poderiam contribuir para o aumento da tensão obtida com o voltímetro?} \\

    Não, os elétrons absorvem os fótons com a frequência de corte. Isso se explica pela equação do efeito fotoelétrico, a energia cinética máxima de um elétron ejetado é a energia do fóton ($hf$) menos a energia gasta para escapar do material (função trabalho). Então, compreende-se que o os fótons ficariam próximos ao átomo de origem e, consequentemente, seriam reabsorvidos. \\
    
\end{enumerate}


\section{Conclusão}
A prática experimental possibilitou determinar aproximadamente
a constante de Planck, a análise da função trabalho e a identificação da frequência de corte da radiação incidente. Apesar de limitações instrumentais e de algumas discrepâncias em relação ao material da fotocélula, o estudo cumpriu seu objetivo didático, demonstrando a validade do modelo quântico da luz.

\begin{thebibliography}{99}
    \bibitem{ceschin} A. Ceschin, \textit{Materiais Elétricos e Magnéticos}. Brasília, DF, 226 p.
    \bibitem{crc} W. M. Haynes (ed.), \textit{CRC Handbook of Chemistry and Physics}, 97ª ed. Boca Raton: CRC Press, 2016.
\end{thebibliography}

\end{document}
